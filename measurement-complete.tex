\documentclass[12pt,a4paper]{article}
% Essential packages for XeLaTeX
\usepackage{fontspec}
\usepackage{unicode-math}
\usepackage[british]{babel}
\usepackage[margin=2.5cm,top=2cm,bottom=2.5cm]{geometry}
\usepackage{xcolor}
\usepackage[most]{tcolorbox}
\usepackage{amsmath,amssymb}
\usepackage{graphicx}
\usepackage{enumitem}
\usepackage{tikz}
\usepackage{pgfplots}
\usepackage{array}
\usepackage{booktabs}
\usepackage{multicol}
\usepackage{fancyhdr}
\usepackage{lastpage}

% TikZ libraries
\usetikzlibrary{arrows.meta,positioning,shapes.geometric,decorations.pathreplacing,calc,patterns}
\pgfplotsset{compat=1.18}

% Page style
\pagestyle{fancy}
\fancyhf{}
\fancyhead[L]{\small Form 2: Measurement and Time}
\fancyhead[R]{\small Page \thepage\ of \pageref{LastPage}}
\renewcommand{\headrulewidth}{0.4pt}

% Colour definitions (British spelling!)
\definecolor{primaryblue}{RGB}{41,128,185}
\definecolor{successgreen}{RGB}{39,174,96}
\definecolor{warningred}{RGB}{231,76,60}
\definecolor{tipyellow}{RGB}{241,196,15}
\definecolor{infopurple}{RGB}{142,68,173}
\definecolor{lightgrey}{RGB}{236,240,241}
\definecolor{darkgrey}{RGB}{52,73,94}

% Custom box environments
\newtcolorbox{conceptbox}[2][]{
enhanced,
title={#2},
colback=primaryblue!8,
colframe=primaryblue!80!black,
coltitle=white,
fonttitle=\bfseries\large,
boxrule=1pt,
arc=3mm,
attach boxed title to top left={yshift=-2.5mm, xshift=5mm},
boxed title style={colback=primaryblue!80!black,arc=2mm,boxrule=0pt},
#1
}

\newtcolorbox{examplebox}[1]{
enhanced,
title={Worked Example: #1},
colback=successgreen!8,
colframe=successgreen!80!black,
coltitle=white,
fonttitle=\bfseries,
boxrule=1pt,
arc=2mm,
attach boxed title to top left={yshift=-2.5mm, xshift=5mm},
boxed title style={colback=successgreen!80!black,arc=2mm,boxrule=0pt}
}

\newtcolorbox{tipbox}[1]{
enhanced,
title={\textbf{💡 Top Tip:} #1},
colback=tipyellow!15,
colframe=tipyellow!80!black,
boxrule=1pt,
arc=2mm,
left=5pt,
right=5pt
}

\newtcolorbox{warningbox}[1]{
enhanced,
title={\textbf{⚠ Common Mistake:} #1},
colback=warningred!10,
colframe=warningred!80!black,
boxrule=1pt,
arc=2mm,
left=5pt,
right=5pt
}

\newtcolorbox{challengebox}[1]{
enhanced,
title={\textbf{🎯 Challenge:} #1},
colback=infopurple!10,
colframe=infopurple!80!black,
boxrule=1pt,
arc=2mm,
left=5pt,
right=5pt
}

\newtcolorbox{exercisebox}[1]{
enhanced,
title={Exercise #1},
colback=lightgrey,
colframe=darkgrey,
coltitle=white,
fonttitle=\bfseries,
boxrule=1pt,
arc=2mm,
attach boxed title to top left={yshift=-2.5mm, xshift=5mm},
boxed title style={colback=darkgrey,arc=2mm,boxrule=0pt}
}

% Custom commands
\newcommand{\degC}{$^\circ$C}
\newcommand{\hint}[1]{\textit{\textcolor{primaryblue}{Hint: #1}}}

\begin{document}

% ================= TITLE PAGE =================
\begin{titlepage}
\begin{center}
\vspace*{1cm}

{\Huge\bfseries Form 2 Mathematics\par}
\vspace{0.5cm}
{\LARGE\color{primaryblue}\bfseries Measurement and Time\par}
\vspace{1cm}
{\Large Student Booklet\par}

\vspace{2cm}

\begin{tikzpicture}
% Clock
\draw[line width=1.5pt] (0,0) circle (2cm);
\foreach \angle in {0,30,...,330}
\draw[line width=1pt] (\angle:1.8cm) -- (\angle:2cm);
\foreach \hour in {1,2,...,12}
\node at ({90-\hour*30}:1.5cm) {\Large\bfseries\hour};
\draw[line width=2pt,-{Stealth[length=3mm]}] (0,0) -- (90:1cm);
\draw[line width=1.5pt,-{Stealth[length=3mm]}] (0,0) -- (30:1.4cm);
\fill (0,0) circle (2pt);

% Ruler
\begin{scope}[xshift=6cm]
\draw[line width=1pt,fill=tipyellow!30] (0,0) rectangle (4,0.5);
\foreach \x in {0,0.5,...,4}
\draw[line width=0.5pt] (\x,0) -- (\x,0.2);
\foreach \x in {0,1,...,4}
\draw[line width=1pt] (\x,0) -- (\x,0.3);
\foreach \x in {0,1,...,4}
\node[below] at (\x,-0.1) {\small\x};
\node at (2,0.8) {\textbf{cm}};
\end{scope}

% Thermometer
\begin{scope}[xshift=-6cm]
\draw[line width=1pt,fill=white] (-0.3,-2) rectangle (0.3,2);
\draw[line width=1pt,fill=warningred!70] (-0.15,-2) rectangle (0.15,0.5);
\draw[line width=1pt,fill=warningred!70] (0,0.5) circle (0.3cm);
\foreach \y in {-2,-1.5,...,2}
\draw[line width=0.5pt] (0.3,\y) -- (0.5,\y);
\node at (1,0) {\degC};
\end{scope}
\end{tikzpicture}

\vspace{2cm}

\begin{tcolorbox}[colback=lightgrey,colframe=darkgrey,width=10cm,arc=2mm]
\begin{tabular}{rl}
\textbf{Name:} & \underline{\hspace{6cm}} \\[8pt]
\textbf{Class:} & \underline{\hspace{6cm}} \\[8pt]
\textbf{Teacher:} & \underline{\hspace{6cm}}
\end{tabular}
\end{tcolorbox}

\vfill

{\large This booklet contains comprehensive lessons and exercises on measurement and time.\par}
\vspace{0.3cm}
{\large Work through each section carefully, showing all your working clearly.\par}

\end{center}
\end{titlepage}

% ================= TABLE OF CONTENTS =================
\tableofcontents
\newpage

% ================= SECTION 1: READING SCALES =================
\section{Reading Scales and Measuring Instruments}

\subsection{Introduction to Reading Scales}

\begin{conceptbox}{Why Do We Need to Read Scales?}
In everyday life, we constantly use measuring instruments:
\begin{itemize}[leftmargin=*]
\item \textbf{At home:} Kitchen scales for baking, thermometers for checking temperature, measuring jugs for liquids
\item \textbf{In shops:} Weighing scales at the checkout, petrol pumps showing litres
\item \textbf{In science:} Thermometers, measuring cylinders, force metres
\item \textbf{In medicine:} Blood pressure monitors, scales for weighing patients
\end{itemize}

All these instruments have \textbf{scales} that we must read accurately. A scale is a series of marks at regular intervals used for measurement.
\end{conceptbox}

\begin{tipbox}{The Golden Rules for Reading Scales}
\begin{enumerate}[leftmargin=*]
\item \textbf{Identify the units} -- What are you measuring? (cm, kg, litres, \degC, etc.)
\item \textbf{Find the labelled numbers} -- These are your reference points
\item \textbf{Count the divisions} -- How many small gaps between the numbers?
\item \textbf{Calculate each division's value} -- Divide the difference by the number of divisions
\item \textbf{Read at eye level} -- Avoid parallax errors (reading from an angle)
\end{enumerate}
\end{tipbox}

\subsection{Understanding Scale Divisions}

\begin{conceptbox}{What is a Division?}
A \textbf{division} is the gap or space between two consecutive marks on a scale. The \textbf{value of one division} tells us what each small step represents.

\begin{center}
\begin{tikzpicture}
% Scale
\draw[line width=1.5pt,|-|] (0,0) -- (10,0);
\foreach \x in {0,2,...,10}
\draw[line width=1.5pt] (\x,-0.3) -- (\x,0.3) node[above] {\x};
\foreach \x in {0,0.5,...,10}
\draw[line width=0.8pt] (\x,-0.15) -- (\x,0.15);

% Annotations
\draw[<->,primaryblue,line width=1pt] (0,-0.8) -- (2,-0.8);
\node[below,primaryblue] at (1,-0.8) {Difference = 2};

\draw[decorate,decoration={brace,amplitude=5pt,mirror},warningred,line width=1pt] (0,-1.3) -- (0.5,-1.3);
\draw[decorate,decoration={brace,amplitude=5pt,mirror},warningred,line width=1pt] (0.5,-1.3) -- (1,-1.3);
\draw[decorate,decoration={brace,amplitude=5pt,mirror},warningred,line width=1pt] (1,-1.3) -- (1.5,-1.3);
\draw[decorate,decoration={brace,amplitude=5pt,mirror},warningred,line width=1pt] (1.5,-1.3) -- (2,-1.3);
\node[below,warningred] at (1,-1.8) {4 divisions};

\node[below,successgreen,font=\bfseries] at (5,-2.3) {Each division = 2 ÷ 4 = 0.5};
\end{tikzpicture}
\end{center}
\end{conceptbox}


\subsection{Step-by-Step Method for Reading Any Scale}

\begin{examplebox}{Reading a Number Line}
\textbf{Question:} Read the value indicated by the arrow.

\begin{center}
\begin{tikzpicture}
% Scale
\draw[line width=1.5pt] (0,0) -- (12,0);
\foreach \x/\label in {0/50, 3/60, 6/70, 9/80, 12/90}
\draw[line width=1.5pt] (\x,-0.3) -- (\x,0.3) node[above] {\label};
\foreach \x in {0,0.6,...,12}
\draw[line width=0.8pt] (\x,-0.15) -- (\x,0.15);

% Arrow
\draw[-{Stealth[length=5mm]},warningred,line width=2pt] (7.2,1) -- (7.2,0.4);
\node[above,warningred,font=\bfseries] at (7.2,1.2) {?};
\end{tikzpicture}
\end{center}

\textbf{Solution:}

\textbf{Step 1:} Find the two labelled numbers either side of the arrow.
\begin{itemize}
\item The arrow is between \textbf{70} and \textbf{80}
\end{itemize}

\textbf{Step 2:} Work out the difference between these numbers.
\begin{align*}
\text{Difference} &= 80 - 70 \\
&= 10
\end{align*}

\textbf{Step 3:} Count the number of divisions (gaps) between the two numbers.
\begin{itemize}
\item There are \textbf{5 divisions} between 70 and 80
\end{itemize}

\textbf{Step 4:} Calculate the value of each small division.
\begin{align*}
\text{Value of one division} &= \frac{\text{Difference}}{\text{Number of divisions}} \\
&= \frac{10}{5} \\
&= 2
\end{align*}

\textbf{Step 5:} Count from the lower labelled number up to the arrow.
\begin{itemize}
\item Starting at 70: \quad 70 $\to$ 72 $\to$ 74 $\to$ 76 $\to$ 78 $\to$ 80
\item The arrow is at the second mark after 70
\item $70 + (2 \times 2) = 70 + 4 = \boxed{74}$
\end{itemize}

\textbf{Answer:} The arrow points to \textbf{74}.
\end{examplebox}

\begin{examplebox}{Reading a Scale with Larger Numbers}
\textbf{Question:} What value does the arrow indicate?

\begin{center}
\begin{tikzpicture}
% Scale
\draw[line width=1.5pt] (0,0) -- (12,0);
\foreach \x/\label in {0/200, 6/400, 12/600}
\draw[line width=1.5pt] (\x,-0.3) -- (\x,0.3) node[above] {\label};
\foreach \x in {0,1.2,...,12}
\draw[line width=0.8pt] (\x,-0.15) -- (\x,0.15);

% Arrow
\draw[-{Stealth[length=5mm]},warningred,line width=2pt] (8.4,1) -- (8.4,0.4);
\node[above,warningred,font=\bfseries] at (8.4,1.2) {?};
\end{tikzpicture}
\end{center}

\textbf{Solution:}

\textbf{Step 1:} The arrow is between \textbf{400} and \textbf{600}.

\textbf{Step 2:} Difference = $600 - 400 = 200$

\textbf{Step 3:} Count divisions: There are \textbf{10 divisions} between 400 and 600.

\textbf{Step 4:} Value of one division = $\frac{200}{10} = 20$

\textbf{Step 5:} Count from 400:
\begin{itemize}
\item The arrow is at the 2nd mark after 400
\item $400 + (2 \times 20) = 400 + 40 = \boxed{440}$
\end{itemize}

\textbf{Answer:} The arrow points to \textbf{440}.
\end{examplebox}

\begin{warningbox}{Avoiding This Common Error}
\textbf{Mistake:} Thinking the arrow is at 420 because it's "about halfway" between 400 and 600.

\textbf{Why it's wrong:} You must count the divisions properly! Each division has a specific value. Don't guess!

\textbf{Correct approach:} Always follow the 5-step method systematically.
\end{warningbox}

\begin{examplebox}{Reading a Scale with Decimals}
\textbf{Question:} Read the measurement shown on this kitchen scale.

\begin{center}
\begin{tikzpicture}
% Scale
\draw[line width=1.5pt] (0,0) -- (10,0);
\foreach \x/\label in {0/1.0, 5/1.5, 10/2.0}
\draw[line width=1.5pt] (\x,-0.3) -- (\x,0.3) node[above] {\label\,kg};
\foreach \x in {0,1,...,10}
\draw[line width=0.8pt] (\x,-0.15) -- (\x,0.15);

% Arrow
\draw[-{Stealth[length=5mm]},warningred,line width=2pt] (7,1) -- (7,0.4);
\node[above,warningred,font=\bfseries] at (7,1.2) {?};
\end{tikzpicture}
\end{center}

\textbf{Solution:}

\textbf{Step 1:} Arrow is between \textbf{1.5\,kg} and \textbf{2.0\,kg}.

\textbf{Step 2:} Difference = $2.0 - 1.5 = 0.5$\,kg

\textbf{Step 3:} There are \textbf{5 divisions} between 1.5 and 2.0.

\textbf{Step 4:} Each division = $\frac{0.5}{5} = 0.1$\,kg

\textbf{Step 5:} From 1.5\,kg, the arrow is at the 2nd mark:
\begin{align*}
1.5 + (2 \times 0.1) &= 1.5 + 0.2 \\
&= \boxed{1.7\text{\,kg}}
\end{align*}

\textbf{Answer:} The scale shows \textbf{1.7\,kg}.
\end{examplebox}


\begin{exercisebox}{1.1 -- Basic Number Lines}
Read the value indicated by each arrow. \hint{Follow the 5-step method for each one!}

\textbf{a)}
\begin{center}
\begin{tikzpicture}[scale=0.8]
\draw[line width=1.5pt] (0,0) -- (12,0);
\foreach \x/\label in {0/10, 6/30, 12/50}
\draw[line width=1.5pt] (\x,-0.3) -- (\x,0.3) node[above] {\label};
\foreach \x in {0,2,...,12}
\draw[line width=0.8pt] (\x,-0.15) -- (\x,0.15);
\draw[-{Stealth[length=4mm]},warningred,line width=2pt] (8,0.8) -- (8,0.4);
\end{tikzpicture}
\end{center}
Answer: \underline{\hspace{3cm}}

\textbf{b)}
\begin{center}
\begin{tikzpicture}[scale=0.8]
\draw[line width=1.5pt] (0,0) -- (12,0);
\foreach \x/\label in {0/100, 4/200, 8/300, 12/400}
\draw[line width=1.5pt] (\x,-0.3) -- (\x,0.3) node[above] {\label};
\foreach \x in {0,0.8,...,12}
\draw[line width=0.8pt] (\x,-0.15) -- (\x,0.15);
\draw[-{Stealth[length=4mm]},warningred,line width=2pt] (2.4,0.8) -- (2.4,0.4);
\end{tikzpicture}
\end{center}
Answer: \underline{\hspace{3cm}}

\textbf{c)}
\begin{center}
\begin{tikzpicture}[scale=0.8]
\draw[line width=1.5pt] (0,0) -- (12,0);
\foreach \x/\label in {0/500, 6/1000, 12/1500}
\draw[line width=1.5pt] (\x,-0.3) -- (\x,0.3) node[above] {\label};
\foreach \x in {0,1.2,...,12}
\draw[line width=0.8pt] (\x,-0.15) -- (\x,0.15);
\draw[-{Stealth[length=4mm]},warningred,line width=2pt] (7.2,0.8) -- (7.2,0.4);
\end{tikzpicture}
\end{center}
Answer: \underline{\hspace{3cm}}

\textbf{d)}
\begin{center}
\begin{tikzpicture}[scale=0.8]
\draw[line width=1.5pt] (0,0) -- (12,0);
\foreach \x/\label in {0/0, 4/20, 8/40, 12/60}
\draw[line width=1.5pt] (\x,-0.3) -- (\x,0.3) node[above] {\label};
\foreach \x in {0,1,...,12}
\draw[line width=0.8pt] (\x,-0.15) -- (\x,0.15);
\draw[-{Stealth[length=4mm]},warningred,line width=2pt] (10,0.8) -- (10,0.4);
\end{tikzpicture}
\end{center}
Answer: \underline{\hspace{3cm}}
\end{exercisebox}

\subsection{Reading Thermometers}

\begin{conceptbox}{Understanding Temperature}
\textbf{Temperature} measures how hot or cold something is. We measure it in \textbf{degrees Celsius} (\degC).

\textbf{Key temperatures to remember:}
\begin{itemize}[leftmargin=*]
\item \textbf{0\degC} -- Water freezes (ice forms)
\item \textbf{100\degC} -- Water boils
\item \textbf{37\degC} -- Normal human body temperature
\item \textbf{20--22\degC} -- Comfortable room temperature
\item \textbf{Below 0\degC} -- Negative temperatures (freezing cold!)
\end{itemize}

Thermometers show positive temperatures \textit{above} zero and negative temperatures \textit{below} zero.
\end{conceptbox}

\begin{examplebox}{Reading a Thermometer (Positive Temperature)}
\textbf{Question:} What temperature is shown on this thermometer?

\begin{center}
\begin{tikzpicture}[scale=0.8]
% Thermometer body
\draw[line width=1.5pt,fill=white] (-0.5,-4) rectangle (0.5,4);
\draw[line width=1.5pt,fill=warningred!70] (-0.25,-4) rectangle (0.25,0.8);
\draw[line width=1.5pt,fill=warningred!70] (0,0.8) circle (0.4cm);

% Scale
\foreach \y/\label in {-4/-10, -3/0, -2/10, -1/20, 0/30, 1/40, 2/50, 3/60, 4/70}
\draw[line width=1pt] (0.5,\y) -- (1,\y) node[right] {\label};
\foreach \y in {-4,-3.8,...,4}
\draw[line width=0.5pt] (0.5,\y) -- (0.7,\y);

% Label
\node at (2,4) {\degC};
\end{tikzpicture}
\end{center}

\textbf{Solution:}

\textbf{Step 1:} The red liquid is between \textbf{30\degC} and \textbf{40\degC}.

\textbf{Step 2:} Difference = $40 - 30 = 10$\degC

\textbf{Step 3:} Count small divisions: There are \textbf{10 divisions}.

\textbf{Step 4:} Each division = $\frac{10}{10} = 1$\degC

\textbf{Step 5:} The liquid reaches 4 marks above 30\degC:
\begin{align*}
30 + (4 \times 1) &= 30 + 4 \\
&= \boxed{34\text{\degC}}
\end{align*}

\textbf{Answer:} The temperature is \textbf{34\degC}.
\end{examplebox}


\begin{examplebox}{Reading a Thermometer (Negative Temperature)}
\textbf{Question:} What temperature is shown?

\begin{center}
\begin{tikzpicture}[scale=0.8]
% Thermometer body
\draw[line width=1.5pt,fill=white] (-0.5,-4) rectangle (0.5,4);
\draw[line width=1.5pt,fill=primaryblue!70] (-0.25,-4) rectangle (0.25,-1.4);
\draw[line width=1.5pt,fill=primaryblue!70] (0,-1.4) circle (0.4cm);

% Scale
\foreach \y/\label in {-4/-20, -3/-10, -2/0, -1/10, 0/20, 1/30, 2/40, 3/50, 4/60}
\draw[line width=1pt] (0.5,\y) -- (1,\y) node[right] {\label};
\foreach \y in {-4,-3.8,...,4}
\draw[line width=0.5pt] (0.5,\y) -- (0.7,\y);

% Label
\node at (2,4) {\degC};
\end{tikzpicture}
\end{center}

\textbf{Solution:}

\textbf{Step 1:} The liquid is between \textbf{$-10$\degC} and \textbf{0\degC}.

\textbf{Step 2:} Difference = $0 - (-10) = 10$\degC

\textbf{Step 3:} There are \textbf{10 divisions}.

\textbf{Step 4:} Each division = $1$\degC

\textbf{Step 5:} The liquid is at 3 marks above $-10$\degC:
\begin{align*}
-10 + (3 \times 1) &= -10 + 3 \\
&= \boxed{-7\text{\degC}}
\end{align*}

\textbf{Answer:} The temperature is \textbf{$-7$\degC} (7 degrees below zero).
\end{examplebox}

\begin{tipbox}{Understanding Negative Temperatures}
\textbf{Remember:}
\begin{itemize}[leftmargin=*]
\item $-5$\degC is \textbf{colder} than $-2$\degC
\item The further below zero, the colder it is
\item Think of it like debt: $-£10$ is worse than $-£5$!
\end{itemize}

\textbf{Number line for negative numbers:}
\begin{center}
\begin{tikzpicture}[scale=0.6]
\draw[<->,line width=1.5pt] (-6,0) -- (6,0);
\foreach \x in {-5,-4,...,5}
\draw[line width=1pt] (\x,-0.2) -- (\x,0.2) node[above] {\x};
\draw[<-,warningred,line width=2pt] (-3,0.8) -- (-3,0.3);
\node[above,warningred] at (-3,0.8) {Colder};
\draw[->,successgreen,line width=2pt] (3,0.8) -- (3,0.3);
\node[above,successgreen] at (3,0.8) {Warmer};
\end{tikzpicture}
\end{center}
\end{tipbox}

\begin{exercisebox}{1.2 -- Reading Thermometers}
Read the temperature shown on each thermometer carefully.

\textbf{a)} 
\begin{center}
\begin{tikzpicture}[scale=0.6]
\draw[line width=1.5pt,fill=white] (-0.5,-3) rectangle (0.5,3);
\draw[line width=1.5pt,fill=warningred!70] (-0.25,-3) rectangle (0.25,1.2);
\draw[line width=1.5pt,fill=warningred!70] (0,1.2) circle (0.4cm);
\foreach \y/\label in {-3/0, -2/10, -1/20, 0/30, 1/40, 2/50, 3/60}
\draw[line width=1pt] (0.5,\y) -- (1,\y) node[right] {\label};
\foreach \y in {-3,-2.8,...,3}
\draw[line width=0.5pt] (0.5,\y) -- (0.7,\y);
\node at (2,3) {\degC};
\end{tikzpicture}
\end{center}
Temperature: \underline{\hspace{3cm}}\degC

\textbf{b)}
\begin{center}
\begin{tikzpicture}[scale=0.6]
\draw[line width=1.5pt,fill=white] (-0.5,-3) rectangle (0.5,3);
\draw[line width=1.5pt,fill=primaryblue!70] (-0.25,-3) rectangle (0.25,-0.6);
\draw[line width=1.5pt,fill=primaryblue!70] (0,-0.6) circle (0.4cm);
\foreach \y/\label in {-3/-20, -2/-10, -1/0, 0/10, 1/20, 2/30, 3/40}
\draw[line width=1pt] (0.5,\y) -- (1,\y) node[right] {\label};
\foreach \y in {-3,-2.8,...,3}
\draw[line width=0.5pt] (0.5,\y) -- (0.7,\y);
\node at (2,3) {\degC};
\end{tikzpicture}
\end{center}
Temperature: \underline{\hspace{3cm}}\degC

\textbf{c)}
\begin{center}
\begin{tikzpicture}[scale=0.6]
\draw[line width=1.5pt,fill=white] (-0.5,-3) rectangle (0.5,3);
\draw[line width=1.5pt,fill=warningred!70] (-0.25,-3) rectangle (0.25,2.5);
\draw[line width=1.5pt,fill=warningred!70] (0,2.5) circle (0.4cm);
\foreach \y/\label in {-3/50, -1.5/60, 0/70, 1.5/80, 3/90}
\draw[line width=1pt] (0.5,\y) -- (1,\y) node[right] {\label};
\foreach \y in {-3,-2.7,...,3}
\draw[line width=0.5pt] (0.5,\y) -- (0.7,\y);
\node at (2,3) {\degC};
\end{tikzpicture}
\end{center}
Temperature: \underline{\hspace{3cm}}\degC
\end{exercisebox}

\begin{challengebox}{Real-World Temperature Challenge}
\textbf{The weather forecast says:}
\begin{itemize}
\item Monday: $-3$\degC
\item Tuesday: $-7$\degC
\item Wednesday: $2$\degC
\end{itemize}

\textbf{Questions:}
\begin{enumerate}[label=\alph*)]
\item Which day was the coldest?
\item Which day was the warmest?
\item What is the temperature difference between Monday and Wednesday?
\end{enumerate}

\textit{Answers at the end of the booklet!}
\end{challengebox}

\newpage

% ================= SECTION 2: UNITS AND CONVERSIONS =================
\section{Units and Conversions}

\subsection{Introduction to the Metric System}

\begin{conceptbox}{The Metric System}
The \textbf{metric system} is used worldwide for measurements. It's based on powers of 10, which makes conversions easy!

\textbf{Main types of measurement:}
\begin{center}
\begin{tabular}{lll}
\toprule
\textbf{Measurement} & \textbf{Units} & \textbf{Symbol} \\
\midrule
Length & millimetre, centimetre, metre, kilometre & mm, cm, m, km \\
Mass & gram, kilogram, tonne & g, kg, t \\
Capacity/Volume & millilitre, litre & ml, l \\
Time & second, minute, hour, day & s, min, h, d \\
\bottomrule
\end{tabular}
\end{center}

\textbf{Why is it called "metric"?} The word comes from "metre", the base unit for length!
\end{conceptbox}

\subsection{Length Conversions}

\begin{conceptbox}{Understanding Length Units}
\textbf{Length} measures distance or how long something is.

\begin{center}
\begin{tikzpicture}[scale=0.9]
% Boxes
\node[draw,fill=tipyellow!30,minimum width=2cm,minimum height=1.2cm] (mm) at (0,0) {\textbf{mm}};
\node[draw,fill=successgreen!30,minimum width=2cm,minimum height=1.2cm] (cm) at (3.5,0) {\textbf{cm}};
\node[draw,fill=primaryblue!30,minimum width=2cm,minimum height=1.2cm] (m) at (7,0) {\textbf{m}};
\node[draw,fill=infopurple!30,minimum width=2cm,minimum height=1.2cm] (km) at (10.5,0) {\textbf{km}};

% Arrows going right (multiply)
\draw[-{Stealth[length=3mm]},line width=1.5pt] (mm) -- (cm) node[midway,above] {$\div 10$};
\draw[-{Stealth[length=3mm]},line width=1.5pt] (cm) -- (m) node[midway,above] {$\div 100$};
\draw[-{Stealth[length=3mm]},line width=1.5pt] (m) -- (km) node[midway,above] {$\div 1000$};

% Arrows going left (divide)
\draw[{Stealth[length=3mm]}-,line width=1.5pt] (mm) -- (cm) node[midway,below] {$\times 10$};
\draw[{Stealth[length=3mm]}-,line width=1.5pt] (cm) -- (m) node[midway,below] {$\times 100$};
\draw[{Stealth[length=3mm]}-,line width=1.5pt] (m) -- (km) node[midway,below] {$\times 1000$};
\end{tikzpicture}
\end{center}

\textbf{Key facts:}
\begin{itemize}[leftmargin=*]
\item 10 millimetres (mm) = 1 centimetre (cm)
\item 100 centimetres (cm) = 1 metre (m)
\item 1000 metres (m) = 1 kilometre (km)
\end{itemize}

\textbf{Real-life examples:}
\begin{itemize}[leftmargin=*]
\item Thickness of a coin $\approx$ 2\,mm
\item Width of your finger $\approx$ 1\,cm
\item Height of a door $\approx$ 2\,m
\item Walking distance to school $\approx$ 1--2\,km
\end{itemize}
\end{conceptbox}

\begin{tipbox}{The "Multiply or Divide?" Trick}
\textbf{Rule:} When you convert to a \textit{smaller} unit, the number gets \textit{bigger}, so you \textbf{multiply}.

When you convert to a \textit{larger} unit, the number gets \textit{smaller}, so you \textbf{divide}.

\textbf{Examples:}
\begin{itemize}[leftmargin=*]
\item m $\to$ cm: metres are bigger than centimetres, so \textbf{multiply} by 100
\item mm $\to$ cm: millimetres are smaller than centimetres, so \textbf{divide} by 10
\end{itemize}
\end{tipbox}


\begin{examplebox}{Converting Metres to Centimetres}
\textbf{Question:} Convert 3.5\,m to centimetres.

\textbf{Solution:}

\textbf{Step 1:} Identify the conversion factor.
\begin{itemize}
\item 1\,m = 100\,cm
\end{itemize}

\textbf{Step 2:} Decide whether to multiply or divide.
\begin{itemize}
\item We're converting to a \textit{smaller} unit (m $\to$ cm)
\item So we \textbf{multiply}
\end{itemize}

\textbf{Step 3:} Perform the calculation.
\begin{align*}
3.5\text{\,m} &= 3.5 \times 100\text{\,cm} \\
&= \boxed{350\text{\,cm}}
\end{align*}

\textbf{Check:} 350\,cm is a bigger number than 3.5\,m -- this makes sense because centimetres are smaller!

\textbf{Answer:} 3.5\,m = \textbf{350\,cm}
\end{examplebox}

\begin{examplebox}{Converting Kilometres to Metres}
\textbf{Question:} Convert 2.45\,km to metres.

\textbf{Solution:}

1\,km = 1000\,m, and we're converting to a smaller unit, so multiply:
\begin{align*}
2.45\text{\,km} &= 2.45 \times 1000\text{\,m} \\
&= \boxed{2450\text{\,m}}
\end{align*}

\textbf{Answer:} 2.45\,km = \textbf{2450\,m}
\end{examplebox}

\begin{examplebox}{Converting Centimetres to Metres}
\textbf{Question:} Convert 750\,cm to metres.

\textbf{Solution:}

100\,cm = 1\,m, and we're converting to a larger unit, so divide:
\begin{align*}
750\text{\,cm} &= 750 \div 100\text{\,m} \\
&= \boxed{7.5\text{\,m}}
\end{align*}

\textbf{Answer:} 750\,cm = \textbf{7.5\,m}
\end{examplebox}

\begin{examplebox}{Two-Step Conversion: mm to m}
\textbf{Question:} Convert 5400\,mm to metres.

\textbf{Solution:}

\textbf{Method 1: Convert in two steps}

Step 1: mm $\to$ cm (divide by 10):
\[5400\text{\,mm} = 5400 \div 10 = 540\text{\,cm}\]

Step 2: cm $\to$ m (divide by 100):
\[540\text{\,cm} = 540 \div 100 = 5.4\text{\,m}\]

\textbf{Method 2: Convert in one step}

Since 1\,m = 1000\,mm, divide by 1000:
\[5400\text{\,mm} = 5400 \div 1000 = 5.4\text{\,m}\]

\textbf{Answer:} 5400\,mm = \textbf{5.4\,m}
\end{examplebox}

\begin{warningbox}{Don't Forget the Decimal Point!}
\textbf{Common mistake:} Writing $3.5 \times 100 = 3.500$ instead of $350$.

\textbf{Remember:}
\begin{itemize}[leftmargin=*]
\item To multiply by 10: Move decimal point 1 place right
\item To multiply by 100: Move decimal point 2 places right
\item To multiply by 1000: Move decimal point 3 places right
\item To divide: Move decimal point left!
\end{itemize}

\textbf{Example:} $3.5 \times 100 = 350.0 = 350$
\end{warningbox}

\begin{exercisebox}{2.1 -- Length Conversions}
Complete these conversions. \hint{Draw the conversion diagram to help you!}

\textbf{Part A: Converting to smaller units (multiply)}
\begin{enumerate}[leftmargin=*]
\item 4\,m = \underline{\hspace{3cm}}\,cm
\item 2.8\,km = \underline{\hspace{3cm}}\,m
\item 6.5\,cm = \underline{\hspace{3cm}}\,mm
\item 0.75\,m = \underline{\hspace{3cm}}\,cm
\end{enumerate}

\textbf{Part B: Converting to larger units (divide)}
\begin{enumerate}[resume,leftmargin=*]
\item 3500\,m = \underline{\hspace{3cm}}\,km
\item 480\,cm = \underline{\hspace{3cm}}\,m
\item 90\,mm = \underline{\hspace{3cm}}\,cm
\item 12500\,m = \underline{\hspace{3cm}}\,km
\end{enumerate}

\textbf{Part C: Challenge conversions}
\begin{enumerate}[resume,leftmargin=*]
\item 6300\,mm = \underline{\hspace{3cm}}\,m
\item 0.085\,km = \underline{\hspace{3cm}}\,cm
\end{enumerate}
\end{exercisebox}

\subsection{Mass Conversions}

\begin{conceptbox}{Understanding Mass}
\textbf{Mass} (or weight) measures how heavy something is.

\begin{center}
\begin{tikzpicture}[scale=0.9]
\node[draw,fill=tipyellow!30,minimum width=2.5cm,minimum height=1.2cm] (g) at (0,0) {\textbf{grams (g)}};
\node[draw,fill=successgreen!30,minimum width=2.5cm,minimum height=1.2cm] (kg) at (4.5,0) {\textbf{kilograms (kg)}};
\node[draw,fill=primaryblue!30,minimum width=2.5cm,minimum height=1.2cm] (t) at (9,0) {\textbf{tonnes (t)}};

\draw[-{Stealth[length=3mm]},line width=1.5pt] (g) -- (kg) node[midway,above] {$\div 1000$};
\draw[-{Stealth[length=3mm]},line width=1.5pt] (kg) -- (t) node[midway,above] {$\div 1000$};

\draw[{Stealth[length=3mm]}-,line width=1.5pt] (g) -- (kg) node[midway,below] {$\times 1000$};
\draw[{Stealth[length=3mm]}-,line width=1.5pt] (kg) -- (t) node[midway,below] {$\times 1000$};
\end{tikzpicture}
\end{center}

\textbf{Key facts:}
\begin{itemize}[leftmargin=*]
\item 1000 grams (g) = 1 kilogram (kg)
\item 1000 kilograms (kg) = 1 tonne (t)
\end{itemize}

\textbf{Real-life examples:}
\begin{itemize}[leftmargin=*]
\item Paper clip $\approx$ 1\,g
\item Bag of sugar = 1\,kg
\item Small car $\approx$ 1\,t
\end{itemize}
\end{conceptbox}

\begin{examplebox}{Converting Kilograms to Grams}
\textbf{Question:} A recipe needs 2.5\,kg of flour. How many grams is this?

\textbf{Solution:}

1\,kg = 1000\,g, and we're converting to a smaller unit, so multiply:
\begin{align*}
2.5\text{\,kg} &= 2.5 \times 1000\text{\,g} \\
&= \boxed{2500\text{\,g}}
\end{align*}

\textbf{Answer:} 2.5\,kg = \textbf{2500\,g}
\end{examplebox}

\begin{examplebox}{Converting Grams to Kilograms}
\textbf{Question:} A parcel weighs 3750\,g. What is this in kilograms?

\textbf{Solution:}

1000\,g = 1\,kg, and we're converting to a larger unit, so divide:
\begin{align*}
3750\text{\,g} &= 3750 \div 1000\text{\,kg} \\
&= \boxed{3.75\text{\,kg}}
\end{align*}

\textbf{Answer:} 3750\,g = \textbf{3.75\,kg}
\end{examplebox}

\begin{exercisebox}{2.2 -- Mass Conversions}
Complete these conversions.

\begin{enumerate}[leftmargin=*]
\item 4.2\,kg = \underline{\hspace{3cm}}\,g
\item 5600\,g = \underline{\hspace{3cm}}\,kg
\item 0.85\,kg = \underline{\hspace{3cm}}\,g
\item 12450\,g = \underline{\hspace{3cm}}\,kg
\item 3.5\,t = \underline{\hspace{3cm}}\,kg
\item 7250\,kg = \underline{\hspace{3cm}}\,t
\end{enumerate}
\end{exercisebox}

\subsection{Volume and Capacity Conversions}

\begin{conceptbox}{Understanding Volume and Capacity}
\textbf{Volume} or \textbf{capacity} measures how much liquid a container can hold.

\begin{center}
\begin{tikzpicture}[scale=0.9]
\node[draw,fill=primaryblue!30,minimum width=3cm,minimum height=1.2cm] (ml) at (0,0) {\textbf{millilitres (ml)}};
\node[draw,fill=successgreen!30,minimum width=3cm,minimum height=1.2cm] (l) at (5,0) {\textbf{litres (l)}};

\draw[-{Stealth[length=3mm]},line width=1.5pt] (ml) -- (l) node[midway,above] {$\div 1000$};
\draw[{Stealth[length=3mm]}-,line width=1.5pt] (ml) -- (l) node[midway,below] {$\times 1000$};
\end{tikzpicture}
\end{center}

\textbf{Key fact:}
\begin{itemize}[leftmargin=*]
\item 1000 millilitres (ml) = 1 litre (l)
\end{itemize}

\textbf{Real-life examples:}
\begin{itemize}[leftmargin=*]
\item Teaspoon $\approx$ 5\,ml
\item Can of drink = 330\,ml
\item Bottle of water = 500\,ml or 1\,l
\item Large bottle of fizzy drink = 2\,l
\end{itemize}
\end{conceptbox}


\begin{examplebox}{Converting Litres to Millilitres}
\textbf{Question:} A jug contains 2.3\,l of juice. How many millilitres is this?

\textbf{Solution:}

1\,l = 1000\,ml, converting to smaller unit, so multiply:
\begin{align*}
2.3\text{\,l} &= 2.3 \times 1000\text{\,ml} \\
&= \boxed{2300\text{\,ml}}
\end{align*}

\textbf{Answer:} 2.3\,l = \textbf{2300\,ml}
\end{examplebox}

\begin{examplebox}{Converting Millilitres to Litres}
\textbf{Question:} A bottle contains 4750\,ml of water. How many litres is this?

\textbf{Solution:}

1000\,ml = 1\,l, converting to larger unit, so divide:
\begin{align*}
4750\text{\,ml} &= 4750 \div 1000\text{\,l} \\
&= \boxed{4.75\text{\,l}}
\end{align*}

\textbf{Answer:} 4750\,ml = \textbf{4.75\,l}
\end{examplebox}

\begin{exercisebox}{2.3 -- Volume/Capacity Conversions}
Complete these conversions.

\begin{enumerate}[leftmargin=*]
\item 3.6\,l = \underline{\hspace{3cm}}\,ml
\item 2800\,ml = \underline{\hspace{3cm}}\,l
\item 0.75\,l = \underline{\hspace{3cm}}\,ml
\item 6250\,ml = \underline{\hspace{3cm}}\,l
\item 1.05\,l = \underline{\hspace{3cm}}\,ml
\item 950\,ml = \underline{\hspace{3cm}}\,l
\end{enumerate}
\end{exercisebox}

\begin{challengebox}{Mixed Units Challenge}
\textbf{Real-world problem:}

Sarah is making fruit punch for a party. She uses:
\begin{itemize}
\item 2.5\,l of orange juice
\item 1500\,ml of lemonade
\item 750\,ml of pineapple juice
\end{itemize}

\textbf{Questions:}
\begin{enumerate}[label=\alph*)]
\item Convert all measurements to millilitres
\item How many millilitres of fruit punch does she make in total?
\item Express your answer in litres
\end{enumerate}

\textit{Answers at the end of the booklet!}
\end{challengebox}

\newpage

% ================= SECTION 3: TIME =================
\section{Time}

\subsection{Understanding Time}

\begin{conceptbox}{Why Do We Measure Time?}
Time helps us organise our day and know when things happen. We use different units depending on what we're measuring:

\begin{center}
\begin{tabular}{ll}
\toprule
\textbf{Unit} & \textbf{When We Use It} \\
\midrule
Seconds & Very short events (running 100\,m) \\
Minutes & Short activities (cooking, travelling) \\
Hours & Longer activities (school day, sleeping) \\
Days & Events spread over time (holidays) \\
\bottomrule
\end{tabular}
\end{center}

\textbf{Key conversions:}
\begin{itemize}[leftmargin=*]
\item 60 seconds = 1 minute
\item 60 minutes = 1 hour
\item 24 hours = 1 day
\end{itemize}
\end{conceptbox}

\subsection{12-Hour and 24-Hour Time}

\begin{conceptbox}{Two Ways to Tell the Time}
There are \textbf{two systems} for writing time:

\textbf{1. 12-Hour Time} uses \textbf{am} and \textbf{pm}:
\begin{itemize}[leftmargin=*]
\item \textbf{am} = \textit{ante meridiem} (Latin: "before midday")
\item Used from midnight (12:00\,am) until noon (12:00\,pm)
\item \textbf{pm} = \textit{post meridiem} (Latin: "after midday")
\item Used from noon (12:00\,pm) until midnight (12:00\,am)
\end{itemize}

\textbf{2. 24-Hour Time} counts from 00:00 to 23:59:
\begin{itemize}[leftmargin=*]
\item Also called "military time" or "railway time"
\item No am/pm needed!
\item Used on timetables, in hospitals, by the military
\end{itemize}

\begin{center}
\begin{tikzpicture}[scale=0.7]
% Clock face
\draw[line width=2pt] (0,0) circle (3cm);
\foreach \angle/\time in {90/12, 60/1, 30/2, 0/3, 330/4, 300/5, 270/6, 240/7, 210/8, 180/9, 150/10, 120/11}
\node at (\angle:2.5cm) {\Large\textbf{\time}};
\foreach \angle in {0,30,...,330}
\draw[line width=1pt] (\angle:2.8cm) -- (\angle:3cm);

% Labels
\node[primaryblue] at (0,1) {\textbf{am}};
\node[warningred] at (0,-1) {\textbf{pm}};
\end{tikzpicture}
\end{center}
\end{conceptbox}

\begin{tipbox}{Understanding Midnight and Noon}
\textbf{Tricky times!}
\begin{itemize}[leftmargin=*]
\item \textbf{Midnight} = 12:00\,am = 00:00 (start of the day)
\item \textbf{Noon/Midday} = 12:00\,pm = 12:00 (middle of the day)
\end{itemize}

\textbf{Memory trick:} 
\begin{itemize}[leftmargin=*]
\item "am" sounds like "morning" -- use it for morning times!
\item "pm" sounds like "afternoon/evening" -- use it after noon!
\end{itemize}
\end{tipbox}

\begin{conceptbox}{Conversion Table: 12-Hour ↔ 24-Hour}
\begin{center}
\begin{tabular}{cc|cc}
\toprule
\textbf{12-Hour} & \textbf{24-Hour} & \textbf{12-Hour} & \textbf{24-Hour} \\
\midrule
12:00\,am & 00:00 & 12:00\,pm & 12:00 \\
1:00\,am & 01:00 & 1:00\,pm & 13:00 \\
2:00\,am & 02:00 & 2:00\,pm & 14:00 \\
3:00\,am & 03:00 & 3:00\,pm & 15:00 \\
6:00\,am & 06:00 & 6:00\,pm & 18:00 \\
9:00\,am & 09:00 & 9:00\,pm & 21:00 \\
11:00\,am & 11:00 & 11:00\,pm & 23:00 \\
\bottomrule
\end{tabular}
\end{center}
\end{conceptbox}

\subsection{Converting Between Time Formats}

\begin{examplebox}{Converting 12-Hour to 24-Hour Time (am)}
\textbf{Question:} Convert 7:30\,am to 24-hour time.

\textbf{Solution:}

\textbf{Rule for am times:} Keep the same, but write with leading zero if needed.
\begin{itemize}
\item 7:30\,am is in the morning (am)
\item In 24-hour time: \boxed{07:30}
\end{itemize}

\textbf{Answer:} 7:30\,am = \textbf{07:30}
\end{examplebox}

\begin{examplebox}{Converting 12-Hour to 24-Hour Time (pm)}
\textbf{Question:} Convert 4:15\,pm to 24-hour time.

\textbf{Solution:}

\textbf{Rule for pm times (except 12:00\,pm):} Add 12 to the hour.
\begin{align*}
\text{Hour: } 4 + 12 &= 16 \\
\text{Minutes: } &= 15 \text{ (stay the same)}
\end{align*}

\textbf{Answer:} 4:15\,pm = \textbf{16:15}
\end{examplebox}


\begin{examplebox}{Converting 12-Hour to 24-Hour Time (Special Cases)}
\textbf{Question:} Convert these times to 24-hour time:
\begin{enumerate}[label=\alph*)]
\item 12:30\,pm (lunchtime)
\item 12:45\,am (just after midnight)
\end{enumerate}

\textbf{Solution:}

\textbf{a)} 12:30\,pm
\begin{itemize}
\item This is just after noon
\item \textbf{Special rule:} 12:00\,pm stays as 12:00 in 24-hour time
\item Answer: \boxed{12:30}
\end{itemize}

\textbf{b)} 12:45\,am
\begin{itemize}
\item This is just after midnight
\item \textbf{Special rule:} 12:00\,am becomes 00:00 in 24-hour time
\item Answer: \boxed{00:45}
\end{itemize}
\end{examplebox}

\begin{examplebox}{Converting 24-Hour to 12-Hour Time}
\textbf{Question:} Convert these 24-hour times to 12-hour time:
\begin{enumerate}[label=\alph*)]
\item 15:40
\item 08:25
\item 23:50
\end{enumerate}

\textbf{Solution:}

\textbf{a)} 15:40
\begin{itemize}
\item 15 is greater than 12, so it's a pm time
\item Subtract 12: $15 - 12 = 3$
\item Answer: \boxed{3:40\text{\,pm}}
\end{itemize}

\textbf{b)} 08:25
\begin{itemize}
\item 08 is less than 12, so it's an am time
\item Just remove the leading zero
\item Answer: \boxed{8:25\text{\,am}}
\end{itemize}

\textbf{c)} 23:50
\begin{itemize}
\item 23 is greater than 12, so it's a pm time
\item Subtract 12: $23 - 12 = 11$
\item Answer: \boxed{11:50\text{\,pm}}
\end{itemize}
\end{examplebox}

\begin{warningbox}{Don't Make This Mistake!}
\textbf{Wrong:} 15:00 = 15:00\,pm

\textbf{Why it's wrong:} You can't mix 24-hour and 12-hour formats!

\textbf{Correct:} 15:00 = 3:00\,pm (subtract 12 from the hour)
\end{warningbox}

\begin{exercisebox}{3.1 -- Time Conversions Part A}
Convert these 12-hour times to 24-hour time.

\begin{enumerate}[leftmargin=*]
\item 6:00\,am = \underline{\hspace{3cm}}
\item 2:30\,pm = \underline{\hspace{3cm}}
\item 11:45\,am = \underline{\hspace{3cm}}
\item 8:15\,pm = \underline{\hspace{3cm}}
\item 12:00\,pm (noon) = \underline{\hspace{3cm}}
\item 12:00\,am (midnight) = \underline{\hspace{3cm}}
\item 9:50\,pm = \underline{\hspace{3cm}}
\item 5:35\,am = \underline{\hspace{3cm}}
\end{enumerate}
\end{exercisebox}

\begin{exercisebox}{3.2 -- Time Conversions Part B}
Convert these 24-hour times to 12-hour time.

\begin{enumerate}[leftmargin=*]
\item 14:20 = \underline{\hspace{3cm}}
\item 07:45 = \underline{\hspace{3cm}}
\item 19:30 = \underline{\hspace{3cm}}
\item 22:15 = \underline{\hspace{3cm}}
\item 00:30 = \underline{\hspace{3cm}}
\item 12:00 = \underline{\hspace{3cm}}
\item 16:55 = \underline{\hspace{3cm}}
\item 03:10 = \underline{\hspace{3cm}}
\end{enumerate}
\end{exercisebox}

\subsection{Adding and Subtracting Time}

\begin{conceptbox}{Why Do We Add and Subtract Time?}
We often need to calculate time in everyday life:
\begin{itemize}[leftmargin=*]
\item "If I leave at 2:30\,pm and the journey takes 1 hour 45 minutes, when will I arrive?"
\item "The film starts at 7:15\,pm and ends at 9:40\,pm. How long is it?"
\end{itemize}

\textbf{Important reminders:}
\begin{itemize}[leftmargin=*]
\item 60 minutes = 1 hour
\item When adding/subtracting, we need to "carry" or "borrow" between hours and minutes
\end{itemize}
\end{conceptbox}

\begin{examplebox}{Adding Time (No Carrying Needed)}
\textbf{Question:} Add 2 hours 25 minutes $+$ 1 hour 30 minutes.

\textbf{Solution:}

\textbf{Step 1:} Add the hours together.
\[2\text{ hours} + 1\text{ hour} = 3\text{ hours}\]

\textbf{Step 2:} Add the minutes together.
\[25\text{ minutes} + 30\text{ minutes} = 55\text{ minutes}\]

\textbf{Step 3:} Check if minutes $<$ 60. If yes, we're done!
\[55 < 60 \checkmark\]

\textbf{Answer:} \boxed{3\text{ hours }55\text{ minutes}}
\end{examplebox}

\begin{examplebox}{Adding Time (Carrying Required)}
\textbf{Question:} Add 2 hours 35 minutes $+$ 1 hour 40 minutes.

\textbf{Solution:}

\textbf{Step 1:} Add hours and minutes separately.
\begin{align*}
\text{Hours: } 2 + 1 &= 3\text{ hours} \\
\text{Minutes: } 35 + 40 &= 75\text{ minutes}
\end{align*}

\textbf{Step 2:} Convert minutes to hours (if $\geq 60$).
\begin{align*}
75\text{ minutes} &= 60\text{ minutes} + 15\text{ minutes} \\
&= 1\text{ hour} + 15\text{ minutes}
\end{align*}

\textbf{Step 3:} Add the extra hour to our total.
\begin{align*}
3\text{ hours} + 1\text{ hour} + 15\text{ minutes} &= 4\text{ hours }15\text{ minutes}
\end{align*}

\textbf{Answer:} \boxed{4\text{ hours }15\text{ minutes}}
\end{examplebox}

\begin{tipbox}{The Column Method for Adding Time}
You can add time like you add regular numbers:

\begin{center}
\begin{tabular}{r|rr}
& \textbf{h} & \textbf{min} \\
\hline
& 2 & 35 \\
$+$ & 1 & 40 \\
\hline
& 3 & 75 \\
\end{tabular}
\quad $\rightarrow$ \quad
\begin{tabular}{r|rr}
& \textbf{h} & \textbf{min} \\
\hline
& 2 & 35 \\
$+$ & 1 & 40 \\
\hline
& 3 & 75 \\
& $+1$ & $-60$ \\
\hline
& 4 & 15 \\
\end{tabular}
\end{center}

When minutes $\geq 60$, subtract 60 and add 1 to hours!
\end{tipbox}


\begin{examplebox}{Subtracting Time (No Borrowing Needed)}
\textbf{Question:} Subtract 5 hours 40 minutes $-$ 2 hours 20 minutes.

\textbf{Solution:}

\textbf{Step 1:} Subtract hours and minutes separately.
\begin{align*}
\text{Hours: } 5 - 2 &= 3\text{ hours} \\
\text{Minutes: } 40 - 20 &= 20\text{ minutes}
\end{align*}

\textbf{Answer:} \boxed{3\text{ hours }20\text{ minutes}}
\end{examplebox}

\begin{examplebox}{Subtracting Time (Borrowing Required)}
\textbf{Question:} Subtract 5 hours 20 minutes $-$ 2 hours 45 minutes.

\textbf{Solution:}

\textbf{Step 1:} Try to subtract minutes.
\[20 - 45 = ?\]
We can't do this! 20 is smaller than 45.

\textbf{Step 2:} Borrow 1 hour from the hours column.
\begin{itemize}
\item 5 hours 20 minutes becomes 4 hours 80 minutes
\item (We converted 1 hour to 60 minutes: $20 + 60 = 80$)
\end{itemize}

\textbf{Step 3:} Now subtract.
\begin{align*}
\text{Hours: } 4 - 2 &= 2\text{ hours} \\
\text{Minutes: } 80 - 45 &= 35\text{ minutes}
\end{align*}

\textbf{Answer:} \boxed{2\text{ hours }35\text{ minutes}}
\end{examplebox}

\begin{warningbox}{Common Borrowing Mistake}
\textbf{Wrong:} When borrowing, adding 10 instead of 60 to the minutes.

\textbf{Example:} 5 h 20 min $\rightarrow$ 4 h 30 min ✗

\textbf{Correct:} 5 h 20 min $\rightarrow$ 4 h 80 min ✓

\textbf{Remember:} 1 hour = \textbf{60} minutes, not 10!
\end{warningbox}

\begin{examplebox}{Real-World Time Problem}
\textbf{Question:} A train leaves at 14:25 and the journey takes 2 hours 50 minutes. What time does it arrive?

\textbf{Solution:}

\textbf{Method 1: Add the times}
\begin{align*}
\text{Start time: } &14:25 \\
\text{Add: } &+ 2\text{ h }50\text{ min} \\
\hline
\text{Hours: } &14 + 2 = 16 \\
\text{Minutes: } &25 + 50 = 75\text{ min}
\end{align*}

Convert: $75\text{ min} = 1\text{ h }15\text{ min}$

So: $16:00 + 1:15 = 17:15$

\textbf{Answer:} The train arrives at \boxed{17:15} (or 5:15\,pm).
\end{examplebox}

\begin{exercisebox}{3.3 -- Adding Time}
Calculate the following. Show your working!

\begin{enumerate}[leftmargin=*]
\item 3 h 20 min $+$ 2 h 15 min = \underline{\hspace{4cm}}
\item 1 h 45 min $+$ 2 h 30 min = \underline{\hspace{4cm}}
\item 5 h 50 min $+$ 3 h 25 min = \underline{\hspace{4cm}}
\item 4 h 35 min $+$ 1 h 55 min = \underline{\hspace{4cm}}
\end{enumerate}
\end{exercisebox}

\begin{exercisebox}{3.4 -- Subtracting Time}
Calculate the following. Show your working!

\begin{enumerate}[leftmargin=*]
\item 6 h 40 min $-$ 3 h 20 min = \underline{\hspace{4cm}}
\item 5 h 15 min $-$ 2 h 35 min = \underline{\hspace{4cm}}
\item 8 h 10 min $-$ 3 h 45 min = \underline{\hspace{4cm}}
\item 7 h 05 min $-$ 4 h 50 min = \underline{\hspace{4cm}}
\end{enumerate}
\end{exercisebox}

\begin{exercisebox}{3.5 -- Time Word Problems}
Solve these problems using addition or subtraction of time.

\textbf{1.} A film starts at 19:30 and lasts 2 hours 25 minutes. What time does it finish?

Answer: \underline{\hspace{4cm}}

\textbf{2.} Sophie leaves home at 08:15 and arrives at school at 08:55. How long does her journey take?

Answer: \underline{\hspace{4cm}}

\textbf{3.} A flight departs at 06:45 and arrives at 10:20. How long is the flight?

Answer: \underline{\hspace{4cm}}

\textbf{4.} Ben spends 1 hour 35 minutes doing homework and 45 minutes playing football. How much time did he spend on both activities?

Answer: \underline{\hspace{4cm}}
\end{exercisebox}

\newpage

% ================= SECTION 4: TIMETABLES =================
\section{Timetables}

\subsection{Introduction to Timetables}

\begin{conceptbox}{What is a Timetable?}
A \textbf{timetable} is a table that shows when transport (buses, trains, aeroplanes) departs from and arrives at different locations.

\textbf{Key vocabulary:}
\begin{itemize}[leftmargin=*]
\item \textbf{Depart/Departure} -- leaving a place
\item \textbf{Arrive/Arrival} -- reaching a place
\item \textbf{Journey time/Duration} -- how long a trip takes
\item \textbf{Stop/Station} -- a place where the transport stops
\end{itemize}

\textbf{Why are timetables important?}
\begin{itemize}[leftmargin=*]
\item Plan journeys to arrive on time
\item Work out how long a journey will take
\item Decide which bus/train to catch
\end{itemize}
\end{conceptbox}

\subsection{Reading a Simple Timetable}

\begin{examplebox}{Understanding a Bus Timetable}
\textbf{Question:} Study this bus timetable and answer the questions.

\begin{center}
\begin{tabular}{l|cccc}
\toprule
\textbf{Stop} & \textbf{Bus 1} & \textbf{Bus 2} & \textbf{Bus 3} & \textbf{Bus 4} \\
\midrule
Town Centre & 08:00 & 09:15 & 10:30 & 11:45 \\
Library & 08:12 & 09:27 & 10:42 & 11:57 \\
Hospital & 08:25 & 09:40 & 10:55 & 12:10 \\
School & 08:35 & 09:50 & 11:05 & 12:20 \\
Shopping Centre & 08:50 & 10:05 & 11:20 & 12:35 \\
\bottomrule
\end{tabular}
\end{center}

\textbf{a)} What time does Bus 2 depart from Town Centre?

\textbf{b)} If I catch Bus 3 from the Library, what time will I arrive at School?

\textbf{c)} How long does the journey take from Town Centre to Shopping Centre on Bus 1?

\textbf{Solution:}

\textbf{a)} Find the Bus 2 column and Town Centre row:
\[\text{Answer: } \boxed{09:15}\]

\textbf{b)} Find Bus 3 column:
\begin{itemize}
\item Library (depart): 10:42
\item School (arrive): 11:05
\item Answer: \boxed{11:05}
\end{itemize}

\textbf{c)} Journey time calculation:
\begin{align*}
\text{Depart Town Centre: } &08:00 \\
\text{Arrive Shopping Centre: } &08:50 \\
\text{Journey time: } &08:50 - 08:00 = \boxed{50\text{ minutes}}
\end{align*}
\end{examplebox}


\begin{tipbox}{How to Use a Timetable Effectively}
\textbf{Step-by-step approach:}
\begin{enumerate}[leftmargin=*]
\item Identify your \textit{starting point} (departure location)
\item Identify your \textit{destination} (arrival location)
\item Find the correct \textit{column} (which bus/train)
\item Read across to find \textit{departure time}
\item Read down to find \textit{arrival time}
\item Calculate \textit{journey time} by subtracting
\end{enumerate}
\end{tipbox}

\begin{examplebox}{Working Backwards from an Arrival Time}
\textbf{Question:} I need to be at the Hospital by 10:00. What is the latest bus I can catch from Town Centre?

\textbf{Bus timetable:}
\begin{center}
\begin{tabular}{l|cccc}
\toprule
\textbf{Stop} & \textbf{Bus 1} & \textbf{Bus 2} & \textbf{Bus 3} & \textbf{Bus 4} \\
\midrule
Town Centre & 08:00 & 09:15 & 10:30 & 11:45 \\
Hospital & 08:25 & 09:40 & 10:55 & 12:10 \\
\bottomrule
\end{tabular}
\end{center}

\textbf{Solution:}

\textbf{Step 1:} Look at Hospital arrival times: 08:25, 09:40, 10:55, 12:10

\textbf{Step 2:} Which arrives \textit{before} 10:00?
\begin{itemize}
\item 08:25 ✓ (too early though!)
\item 09:40 ✓ (this is the latest before 10:00)
\item 10:55 ✗ (too late)
\item 12:10 ✗ (too late)
\end{itemize}

\textbf{Step 3:} Bus 2 arrives at 09:40. When does it depart Town Centre?
\begin{itemize}
\item Answer: \boxed{09:15}
\end{itemize}

\textbf{Answer:} The latest bus from Town Centre is \textbf{Bus 2 at 09:15}.
\end{examplebox}

\subsection{Complex Timetable Problems}

\begin{examplebox}{Multiple Stops and Connections}
\textbf{Question:} Using the timetable below, answer these questions:

\begin{center}
\begin{tabular}{l|ccccc}
\toprule
\textbf{Stop} & \textbf{Bus A} & \textbf{Bus B} & \textbf{Bus C} & \textbf{Bus D} & \textbf{Bus E} \\
\midrule
Station & 07:15 & 08:30 & 09:45 & 11:00 & 12:15 \\
Park Road & 07:28 & 08:43 & 09:58 & 11:13 & 12:28 \\
High Street & 07:35 & 08:50 & 10:05 & 11:20 & 12:35 \\
College & 07:47 & 09:02 & 10:17 & 11:32 & 12:47 \\
Museum & 08:00 & 09:15 & 10:30 & 11:45 & 13:00 \\
Beach & 08:20 & 09:35 & 10:50 & 12:05 & 13:20 \\
\bottomrule
\end{tabular}
\end{center}

\textbf{a)} What time does Bus C arrive at the Beach?

\textbf{b)} How long does the complete journey from Station to Beach take on Bus B?

\textbf{c)} I need to be at College by 10:00. What is the latest bus I can catch from High Street?

\textbf{Solution:}

\textbf{a)} Look at Bus C column, Beach row: \boxed{10:50}

\textbf{b)} Bus B journey time:
\begin{align*}
\text{Depart Station: } &08:30 \\
\text{Arrive Beach: } &09:35 \\
\text{Journey time: } &09:35 - 08:30
\end{align*}

Calculate:
\begin{align*}
9\text{ h }35\text{ min} - 8\text{ h }30\text{ min} &= (9-8)\text{ h } + (35-30)\text{ min} \\
&= 1\text{ h }5\text{ min} \\
&= \boxed{1\text{ hour }5\text{ minutes}}
\end{align*}

\textbf{c)} College arrival times: 07:47, 09:02, 10:17, 11:32, 12:47

Latest before 10:00 is 09:02 (Bus B).

But we're catching from High Street, not Station!

Look at Bus B, High Street row: \boxed{08:50}

\textbf{Answer:} Catch Bus B at 08:50 from High Street.
\end{examplebox}

\begin{exercisebox}{4.1 -- Bus Timetable Practice}
Use this timetable to answer the questions:

\begin{center}
\begin{tabular}{l|ccccc}
\toprule
\textbf{Stop} & \textbf{Bus 10} & \textbf{Bus 20} & \textbf{Bus 30} & \textbf{Bus 40} & \textbf{Bus 50} \\
\midrule
City Centre & 06:00 & 07:30 & 09:00 & 10:30 & 12:00 \\
Park & 06:15 & 07:45 & 09:15 & 10:45 & 12:15 \\
University & 06:30 & 08:00 & 09:30 & 11:00 & 12:30 \\
Hospital & 06:50 & 08:20 & 09:50 & 11:20 & 12:50 \\
Airport & 07:20 & 08:50 & 10:20 & 11:50 & 13:20 \\
\bottomrule
\end{tabular}
\end{center}

\textbf{1.} What time does Bus 20 depart from City Centre?

Answer: \underline{\hspace{4cm}}

\textbf{2.} If I catch Bus 30 from the University, when will I arrive at the Airport?

Answer: \underline{\hspace{4cm}}

\textbf{3.} How long does it take to travel from Park to Hospital on Bus 40?

Answer: \underline{\hspace{4cm}}

\textbf{4.} I need to be at the Hospital by 10:00. What is the latest bus I can catch from City Centre?

Answer: \underline{\hspace{4cm}}

\textbf{5.} What is the total journey time from City Centre to Airport on Bus 10?

Answer: \underline{\hspace{4cm}}
\end{exercisebox}

\begin{challengebox}{Timetable Challenge}
\textbf{Multi-step problem:}

Using the timetable from Exercise 4.1:

James needs to meet his friend at the Airport at 11:00. He lives near the Park stop.

\begin{enumerate}[label=\alph*)]
\item Which bus should he catch from the Park?
\item What time should he leave Park?
\item What time will he arrive at the Airport?
\item How early will he be?
\end{enumerate}

\textit{Hint: Work backwards from 11:00!}
\end{challengebox}

\newpage

% ================= SECTION 5: TIME ZONES =================
\section{Introduction to Time Zones}

\subsection{Why Do We Have Time Zones?}

\begin{conceptbox}{The Earth's Rotation and Time}
The Earth rotates (spins) once every \textbf{24 hours}. As it rotates:
\begin{itemize}[leftmargin=*]
\item Different parts of the Earth face the Sun
\item When it's daytime in one place, it's night-time in another
\item This means different places have different \textit{local times}
\end{itemize}

\begin{center}
\begin{tikzpicture}[scale=0.8]
% Sun
\shade[ball color=tipyellow] (-4,0) circle (0.8cm);
\node at (-4,-1.5) {\textbf{Sun}};

% Earth
\shade[ball color=primaryblue!50] (2,0) circle (1.5cm);

% Day/Night
\fill[black,opacity=0.7] (2,0) -- (2,-1.5) arc (-90:90:1.5cm) -- cycle;
\node[white] at (1,0) {\textbf{Night}};
\node at (3,0) {\textbf{Day}};

% Rays
\foreach \y in {-1,0,1}
\draw[-{Stealth[length=2mm]},tipyellow,line width=1pt] (-3.2,\y*0.5) -- (0.5,\y*0.5);

% Rotation arrow
\draw[-{Stealth[length=3mm]},warningred,line width=2pt] (2,2.5) arc (90:180:1cm);
\node[warningred] at (3.5,2) {\textbf{Rotation}};
\end{tikzpicture}
\end{center}

\textbf{Time zones} divide the world into regions where everyone uses the same time.
\end{conceptbox}

\begin{conceptbox}{Understanding UTC}
\textbf{UTC} stands for \textbf{Coordinated Universal Time}.

\begin{itemize}[leftmargin=*]
\item It's the "reference time" for the whole world
\item Also called \textbf{GMT} (Greenwich Mean Time) -- based on time in London, UK
\item Time zones are described as UTC+1, UTC+2, etc. or UTC-1, UTC-2, etc.
\end{itemize}

\textbf{What do the numbers mean?}
\begin{itemize}[leftmargin=*]
\item UTC+3 = 3 hours \textit{ahead} of London
\item UTC-5 = 5 hours \textit{behind} London
\end{itemize}
\end{conceptbox}

\subsection{Calculating Time in Different Zones}

\begin{examplebox}{Basic Time Zone Calculation}
\textbf{Question:} It is 14:00 in London (UTC). What time is it in Dubai (UTC+4)?

\textbf{Solution:}

\textbf{Step 1:} Identify the time zone difference.
\begin{itemize}
\item Dubai is UTC+4
\item This means Dubai is 4 hours \textit{ahead} of London
\end{itemize}

\textbf{Step 2:} Add 4 hours to London time.
\begin{align*}
14:00 + 4\text{ hours} &= 18:00
\end{align*}

\textbf{Answer:} When it's 14:00 in London, it's \boxed{18:00} in Dubai.
\end{examplebox}

\begin{tipbox}{The Easy Time Zone Rule}
\textbf{UTC+} means the time is \textit{later} (ahead) -- so \textbf{ADD} hours

\textbf{UTC-} means the time is \textit{earlier} (behind) -- so \textbf{SUBTRACT} hours

\textbf{Memory trick:} The + and - signs tell you what to do!
\end{tipbox}

\begin{examplebox}{Time Zone Across Midnight}
\textbf{Question:} It is 21:00 in Los Angeles (UTC-8). What time is it in London (UTC)?

\textbf{Solution:}

\textbf{Step 1:} Los Angeles is UTC-8, meaning it's 8 hours \textit{behind} London.

So London is 8 hours \textit{ahead} of Los Angeles.

\textbf{Step 2:} Add 8 hours to Los Angeles time.
\begin{align*}
21:00 + 8\text{ hours} &= 29:00
\end{align*}

Wait! 29:00 doesn't exist!

\textbf{Step 3:} When we go past 24:00, we go into the next day.
\begin{align*}
29:00 &= 24:00 + 5:00 \\
&= 05:00 \text{ (next day)}
\end{align*}

\textbf{Answer:} When it's 21:00 in Los Angeles, it's \boxed{05:00} the next day in London.
\end{examplebox}

\subsection{Australian Time Zones}

\begin{conceptbox}{Time Zones in Australia}
Australia is a huge country with \textbf{three main time zones}:

\begin{center}
\begin{tabular}{lcc}
\toprule
\textbf{Time Zone} & \textbf{States/Territories} & \textbf{UTC Offset} \\
\midrule
Western Standard Time (WST) & Western Australia & UTC+8 \\
Central Standard Time (CST) & South Australia, NT & UTC+9.5 \\
Eastern Standard Time (EST) & QLD, NSW, VIC, TAS & UTC+10 \\
\bottomrule
\end{tabular}
\end{center}

\textbf{Notice:} Central time is UTC+9.5 -- a half-hour difference!
\end{conceptbox}

\begin{examplebox}{Australian Time Zone Problem}
\textbf{Question:} It is 15:00 in Perth (WST, UTC+8). What time is it in:
\begin{enumerate}[label=\alph*)]
\item Adelaide (CST, UTC+9.5)?
\item Sydney (EST, UTC+10)?
\end{enumerate}

\textbf{Solution:}

\textbf{a)} Adelaide time zone difference:
\begin{align*}
\text{Adelaide: } &\text{UTC+9.5} \\
\text{Perth: } &\text{UTC+8} \\
\text{Difference: } &9.5 - 8 = 1.5\text{ hours}
\end{align*}

Adelaide is 1.5 hours (1 hour 30 minutes) ahead:
\begin{align*}
15:00 + 1:30 &= \boxed{16:30}
\end{align*}

\textbf{b)} Sydney time zone difference:
\begin{align*}
\text{Sydney: } &\text{UTC+10} \\
\text{Perth: } &\text{UTC+8} \\
\text{Difference: } &10 - 8 = 2\text{ hours}
\end{align*}

Sydney is 2 hours ahead:
\begin{align*}
15:00 + 2:00 &= \boxed{17:00}
\end{align*}
\end{examplebox}

\begin{exercisebox}{5.1 -- Time Zone Practice}
Calculate the time in each location.

\textbf{1.} It is 12:00 in London (UTC). What time is it in:
\begin{enumerate}[label=\alph*)]
\item Paris (UTC+1): \underline{\hspace{3cm}}
\item New York (UTC-5): \underline{\hspace{3cm}}
\item Tokyo (UTC+9): \underline{\hspace{3cm}}
\end{enumerate}

\textbf{2.} It is 18:00 in Sydney (UTC+10). What time is it in:
\begin{enumerate}[label=\alph*)]
\item Perth (UTC+8): \underline{\hspace{3cm}}
\item London (UTC): \underline{\hspace{3cm}}
\item Los Angeles (UTC-8): \underline{\hspace{3cm}}
\end{enumerate}

\textbf{3.} It is 09:30 in Adelaide (UTC+9.5). What time is it in Brisbane (UTC+10)?

Answer: \underline{\hspace{4cm}}
\end{exercisebox}

\begin{challengebox}{Time Zone Challenge}
\textbf{Real-world problem:}

Your pen pal in New York (UTC-5) says they will call you at 19:00 their time.

You live in Melbourne, Australia (UTC+10).

\textbf{What time should you expect the call in Melbourne?}

\textit{Hint: Melbourne is 15 hours ahead of New York!}
\end{challengebox}

\newpage

% ================= ANSWERS SECTION =================
\section{Answers}

\subsection{Section 1: Reading Scales}

\textbf{Exercise 1.1:}
\begin{enumerate}[label=\alph*)]
\item 38
\item 160
\item 1100
\item 50
\end{enumerate}

\textbf{Exercise 1.2:}
\begin{enumerate}[label=\alph*)]
\item 42°C
\item −6°C
\item 85°C
\end{enumerate}

\textbf{Challenge (Page 6):}
\begin{enumerate}[label=\alph*)]
\item Tuesday was coldest (−7°C)
\item Wednesday was warmest (2°C)
\item Difference: $2 - (-3) = 2 + 3 = 5$°C
\end{enumerate}

\subsection{Section 2: Units and Conversions}

\textbf{Exercise 2.1:}
\begin{enumerate}
\item 400\,cm
\item 2800\,m
\item 65\,mm
\item 75\,cm
\item 3.5\,km
\item 4.8\,m
\item 9\,cm
\item 12.5\,km
\item 6.3\,m (or 630\,cm)
\item 8500\,cm (or 85\,m)
\end{enumerate}

\textbf{Exercise 2.2:}
\begin{enumerate}
\item 4200\,g
\item 5.6\,kg
\item 850\,g
\item 12.45\,kg
\item 3500\,kg
\item 7.25\,t
\end{enumerate}

\textbf{Exercise 2.3:}
\begin{enumerate}
\item 3600\,ml
\item 2.8\,l
\item 750\,ml
\item 6.25\,l
\item 1050\,ml
\item 0.95\,l
\end{enumerate}

\textbf{Challenge (Page 18):}
\begin{enumerate}[label=\alph*)]
\item Orange juice: 2.5\,l = 2500\,ml; Lemonade: 1500\,ml; Pineapple: 750\,ml
\item Total: $2500 + 1500 + 750 = 4750$\,ml
\item 4750\,ml = 4.75\,l
\end{enumerate}

\subsection{Section 3: Time}

\textbf{Exercise 3.1:}
\begin{enumerate}
\item 06:00
\item 14:30
\item 11:45
\item 20:15
\item 12:00
\item 00:00
\item 21:50
\item 05:35
\end{enumerate}

\textbf{Exercise 3.2:}
\begin{enumerate}
\item 2:20\,pm
\item 7:45\,am
\item 7:30\,pm
\item 10:15\,pm
\item 12:30\,am
\item 12:00\,pm (noon)
\item 4:55\,pm
\item 3:10\,am
\end{enumerate}

\textbf{Exercise 3.3:}
\begin{enumerate}
\item 5 h 35 min
\item 4 h 15 min
\item 9 h 15 min
\item 6 h 30 min
\end{enumerate}

\textbf{Exercise 3.4:}
\begin{enumerate}
\item 3 h 20 min
\item 2 h 40 min
\item 4 h 25 min
\item 2 h 15 min
\end{enumerate}

\textbf{Exercise 3.5:}
\begin{enumerate}
\item 21:55 (or 9:55\,pm)
\item 40 minutes
\item 3 h 35 min
\item 2 h 20 min
\end{enumerate}

\subsection{Section 4: Timetables}

\textbf{Exercise 4.1:}
\begin{enumerate}
\item 07:30
\item 10:20
\item 30 minutes
\item Bus 20 at 07:30
\item 1 h 20 min
\end{enumerate}

\textbf{Challenge (Page 33):}
\begin{enumerate}[label=\alph*)]
\item Bus 40
\item 10:45
\item 11:50
\item 50 minutes early
\end{enumerate}

\subsection{Section 5: Time Zones}

\textbf{Exercise 5.1:}

\textbf{1.}
\begin{enumerate}[label=\alph*)]
\item 13:00
\item 07:00
\item 21:00
\end{enumerate}

\textbf{2.}
\begin{enumerate}[label=\alph*)]
\item 16:00
\item 08:00
\item 00:00 (midnight) same day
\end{enumerate}

\textbf{3.} 10:00

\textbf{Challenge (Page 36):}

Melbourne is UTC+10, New York is UTC-5.

Difference: $10 - (-5) = 15$ hours ahead

New York time: 19:00

Melbourne time: $19:00 + 15:00 = 34:00 = 10:00$ next day

\textbf{Answer:} 10:00 (morning) the next day

\vfill

\begin{center}
\begin{tcolorbox}[colback=successgreen!20,colframe=successgreen!80!black,width=12cm,arc=3mm]
\centering
\Large\textbf{Well Done!}

You've completed the Form 2 Measurement and Time booklet.

Remember to ask your teacher about anything you found difficult.

Keep practising and you'll become an expert at reading scales, converting units, telling the time, and understanding timetables!
\end{tcolorbox}
\end{center}

\end{document}
